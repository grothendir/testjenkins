% Created 2018-12-28 ven. 13:43
\documentclass[11pt]{article}
\usepackage[utf8]{inputenc}
\usepackage[T1]{fontenc}
\usepackage{fixltx2e}
\usepackage{graphicx}
\usepackage{longtable}
\usepackage{float}
\usepackage{wrapfig}
\usepackage{rotating}
\usepackage[normalem]{ulem}
\usepackage{amsmath}
\usepackage{textcomp}
\usepackage{marvosym}
\usepackage{wasysym}
\usepackage{amssymb}
\usepackage{hyperref}
\tolerance=1000
\author{ulysse}
\date{\today}
\title{doc-jenkins}
\hypersetup{
  pdfkeywords={},
  pdfsubject={},
  pdfcreator={Emacs 25.2.2 (Org mode 8.2.10)}}
\begin{document}

\maketitle
\tableofcontents

cf \href{https://linuxize.com/post/how-to-install-jenkins-on-centos-7/}{documentation}
\section{installation}
\label{sec-1}
\subsection{installer java}
\label{sec-1-1}
\begin{verbatim}
$ sudo yum install java
\end{verbatim}
vérifer la version:
\begin{verbatim}
$ java -version
\end{verbatim}
sortie standard:
\begin{verbatim}
openjdk version "1.8.0_191"
OpenJDK Runtime Environment (build 1.8.0_191-b12)
OpenJDK 64-Bit Server VM (build 25.191-b12, mixed mode)
\end{verbatim}
\subsection{installer jenkins}
\label{sec-1-2}
\begin{verbatim}
$ curl --silent --location http://pkg.jenkins-ci.org/redhat-stable/jenkins.repo | sudo tee /etc/yum.repos.d/jenkins.repo
\end{verbatim}

\begin{verbatim}
$ sudo rpm --import https://jenkins-ci.org/redhat/jenkins-ci.org.key
\end{verbatim}

\begin{verbatim}
$ sudo yum install jenkins
\end{verbatim}

\begin{verbatim}
$ sudo systemctl start jenkins
\end{verbatim}

\begin{verbatim}
$ sudo systemctl status jenkins
\end{verbatim}

\begin{verbatim}
$ sudo systemctl enable jenkins
\end{verbatim}

\subsection{configurer le firewall}
\label{sec-1-3}
\begin{verbatim}
sudo firewall-cmd --permanent --zone=public --add-port=8080/tcp
\end{verbatim}

\begin{verbatim}
sudo firewall-cmd --reload
\end{verbatim}

\subsection{setting up jenkins}
\label{sec-1-4}

\section{administration}
\label{sec-2}
\subsection{plugins}
\label{sec-2-1}
\begin{verbatim}
http://<server_ip>:8080/pluginManager/
\end{verbatim}
où <server$_{\text{ip}}$> est l'adresse de votre serveur jenkins

\begin{verbatim}
http://54.38.180.120:8080/pluginManager/
\end{verbatim}

exemples: chercher \emph{python} et \emph{php}, \emph{wordpress}, \emph{ruby}, etc

\subsection{installer pipeline maven integration}
\label{sec-2-2}
chercher le \textbf{plugin} dans l'onglet \emph{disponible}

\subsection{configuration globale des outils}
\label{sec-2-3}
\begin{itemize}
\item jdk --> nécessite la crétion d'un compte oracle
\end{itemize}
\section{jobs}
\label{sec-3}
\subsection{créer un nouveau job}
\label{sec-3-1}
pipeline
\begin{itemize}
\item \emph{sauver}
\item \emph{back to dashboard}
\item \emph{sélectionner un projet}
\item pipeline syntax
\end{itemize}
% Emacs 25.2.2 (Org mode 8.2.10)
\end{document}
